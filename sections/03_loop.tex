% Seção 3: Controle de Loop Fechado

\section{Controle de Loop Fechado no 5G}
\begin{frame}
    \frametitle{O que é Controle de Loop Fechado?}
    \begin{itemize}
        \item \textbf{Definição}: Sistema de automação que ajusta parâmetros da rede em tempo real com base em feedback contínuo.
        \item \textbf{Componentes}: Monitoramento, análise, decisão e execução.
        \item \textbf{Benefícios}: Otimização contínua da rede, melhoria na Qualidade de Serviço (QoS), resposta rápida a falhas.
    \end{itemize}
\end{frame}

\begin{frame}
    \frametitle{Aplicações do Controle de Loop Fechado no 5G}
    \begin{itemize}
        \item \textbf{Self-Organizing Networks (SON)}: Redes que se auto-configuram, otimizam e recuperam.
        \item \textbf{Gerenciamento de Tráfego}: Ajustes automáticos para otimizar o fluxo de dados.
        \item \textbf{Manutenção Preditiva}: Prevenção de falhas com base em análise contínua de dados.
    \end{itemize}
\end{frame}